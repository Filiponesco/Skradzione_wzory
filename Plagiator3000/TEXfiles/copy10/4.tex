\documentclass{article}
\usepackage[utf8]{inputenc}

\begin{document}
Enim ut tellus elementum sagittis. Dignissim suspendisse in est ante in nibh mauris cursus. Nulla facilisi nullam vehicula ipsum a arcu cursus vitae congue. Dolor sit amet consectetur adipiscing elit ut. Fames ac turpis egestas integer eget aliquet. Proin sagittis \begin{math}
	a^n-1=(a-1)(a^{n-1}+a^{n-2}+\cdots+a+1)
\end{math}.\newline

Duis at consectetur lorem $ S_{xx}=\sum_{i=1}^{n}x_i^2 $.\newline

Vitae elementum curabitur vitae nunc sed velit dignissim sodales. Lorem ipsum dolor \( \int\limits_{x\in C}xdx \).\newline

Malesuada fames ac: \[ \sqrt[3]{8}=8^{\frac{1}{3}}=2 \]

Ultrices in iaculis nunc sed augue lacus viverra vitae congue.

$$ x^{2+a} $$

Ac turpis egestas integer eget aliquet nibh praesent. Aliquet bibendum enim facilisis gravida neque convallis. Nascetur ridiculus mus mauris vitae ultricies. Sodales ut etiam sit amet nisl purus in mollis. Urna duis convallis convallis tellus id interdum. Consectetur adipiscing elit ut aliquam purus sit amet. Est ultricies integer quis auctor elit sed. At auctor urna nunc id cursus metus. Id aliquet lectus proin nibh nisl condimentum. Massa sed elementum tempus egestas sed sed risus pretium. Odio eu feugiat pretium nibh ipsum. Adipiscing at in tellus integer feugiat scelerisque.

\begin{displaymath}
	\int\limits_{x\in Z}\! x^{n}\, dx
\end{displaymath}

Lorem ipsum dolor sit amet, consectetur adipiscing elit, sed do eiusmod tempor incididunt ut labore et dolore magna aliqua. Enim ut tellus elementum sagittis vitae. Quis lectus nulla at volutpat diam ut. Lacus suspendisse faucibus interdum posuere lorem. 

\begin{equation}
	P_n\left ( x \right ) = \frac{1 \cdot d^n\left ( x^2-1 \right )^2}{2^n ! \cdot dx^n}
\end{equation}

Lacus luctus accumsan tortor posuere ac ut. Velit ut tortor pretium viverra.

\end{document}
