\documentclass{article}
\begin{document}
	
   Lorem ipsum dolor sit amet, consectetur adipiscing elit, sed do eiusmod tempor incididunt ut labore et dolore magna aliqua. Ut enim ad minim veniam, quis nostrud exercitation ullamco laboris nisi ut aliquip ex ea commodo consequat. Duis aute irure dolor in reprehenderit in voluptate velit esse cillum dolore eu fugiat nulla pariatur. Excepteur sint occaecat cupidatat non proident, sunt in culpa qui officia deserunt mollit anim id est laborum 
   \begin{math}
   U(g_1-g_2)=k\cdot \sqrt{[u(g_1)]^2+[u(g_2)]^2}
    \end{math}.
    \newline
    
    
    Lorem ipsum dolor sit amet, consectetur adipiscing elit, sed do eiusmod tempor incididunt ut labore et dolore magna aliqua. Platea dictumst quisque sagittis purus sit amet volutpat consequat. Volutpat odio facilisis mauris sit amet massa vitae tortor condimentum $ \frac{\sin \alpha}{\sin \beta}=\frac{v_1}{v_2}=n_{12} $.
    \newline
    
    Risus commodo viverra maecenas accumsan lacus vel facilisis volutpat est. Praesent tristique magna sit amet. Tincidunt id aliquet risus feugiat in. Tempor nec feugiat nisl pretium \( u_a(x)=\sqrt{\frac{\sum_{i=1}^{N} (x_i-\overline{x})^2}{N(N-1)}} \).
    \newline
    
    Pharetra magna ac placerat vestibulum lectus mauris ultrices eros in. Etiam dignissim diam quis enim lobortis scelerisque fermentum dui. Interdum posuere lorem ipsum dolor sit. Proin libero nunc consequat interdum. Iaculis nunc sed augue lacus viverra vitae. Ut porttitor leo a diam sollicitudin tempor. Adipiscing elit ut aliquam purus sit amet luctus venenatis. Gravida dictum fusce ut placerat orci nulla pellentesque.
    
    \[ u_c(x)=\sqrt{(u_a)^2+(u_b)^2}  \]
    
    Elementum facilisis leo vel fringilla est ullamcorper eget nulla. Pretium aenean pharetra magna ac placerat vestibulum lectus mauris ultrices. Pulvinar etiam non quam lacus suspendisse faucibus interdum posuere. 
    
    $$ u(T)=\frac{1}{N} \cdot u_a (t_{sr}) $$
    
    Nunc scelerisque viverra mauris in aliquam sem fringilla ut morbi. Mus mauris vitae ultricies leo integer. Sed euismod nisi porta lorem mollis aliquam ut. Ipsum a arcu cursus vitae congue mauris rhoncus aenean vel. Quis risus sed vulputate odio ut enim blandit volutpat. Rhoncus mattis rhoncus urna neque viverra justo. Hendrerit gravida rutrum quisque non tellus orci ac auctor augue. Volutpat est velit egestas dui id ornare arcu. 
    
    \begin{displaymath}
    	a^{3}+1=(a+1)(a^{2}-a+1)
    \end{displaymath}
    
    Quis viverra nibh cras pulvinar mattis nunc sed. Viverra nibh cras pulvinar mattis nunc. Dictum varius duis at consectetur lorem donec. Faucibus in ornare quam viverra orci sagittis eu. Velit ut tortor pretium viverra suspendisse potenti.
    
    
    
    \begin{equation}
	    n=\frac{\sin \frac{1}{2}(\varphi +\delta)}{\sin \frac{1}{2}\varphi}
    \end{equation}
    
    In hac habitasse platea dictumst quisque sagittis purus sit. Imperdiet sed euismod nisi porta lorem mollis. 
    
\end{document}