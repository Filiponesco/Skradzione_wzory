\documentclass{article}
\usepackage[utf8]{inputenc}

\begin{document}
Volutpat lacus laoreet non curabitur gravida arcu ac. Quisque non tellus \begin{math}
	{n \choose k}=\frac{n!}{k!(n-k)}
\end{math}.\newline

Pulvinar pellentesque habitant morbi tristique senectus et netus et. Augue interdum $ a^2-b^2=(a-b)(a+b) $.\newline

Etiam tempor orci eu \( a^3-b^3=(a-b)(a^2-ab+b^2) \).\newline

Venenatis a condimentum vitae sapien pellentesque. Est ullamcorper eget nulla facilisi etiam dignissim diam. Pharetra pharetra massa massa ultricies mi quis hendrerit dolor magna. Facilisis sed odio morbi quis commodo.

\[ a^n-1=(a-1)(a^{n-1}+a^{n-2}+\cdots+a+1) \]

Placerat in egestas erat imperdiet sed. Aliquam faucibus purus in massa. In arcu cursus euismod quis viverra nibh. Eget mauris pharetra et ultrices neque ornare aenean euismod elementum.

$$ a_n=a_1+(n-1)r $$

Dui faucibus in ornare quam viverra orci sagittis. Integer vitae justo eget magna fermentum iaculis eu non. Convallis convallis tellus id interdum velit laoreet id donec ultrices. 

\begin{displaymath}
	f(x)=ax^2+bx+c
\end{displaymath}

Turpis egestas pretium aenean pharetra. Netus et malesuada fames ac turpis egestas maecenas pharetra convallis. Aliquet enim tortor at auctor urna nunc id. Malesuada proin libero nunc consequat. Id eu nisl nunc mi ipsum faucibus vitae.

\begin{equation}
	x_1=\frac{-b-\sqrt{\Delta}}{2a},
	x_2=\frac{-b+\sqrt{\Delta}}{2a}
\end{equation}

Pretium nibh ipsum consequat nisl vel pretium lectus quam. Fringilla phasellus faucibus scelerisque eleifend donec pretium vulputate. Aliquet risus feugiat in ante. Neque aliquam vestibulum morbi blandit cursus. Enim nulla aliquet porttitor lacus luctus accumsan tortor posuere. Elit ut aliquam purus sit amet luctus venenatis lectus magna. Molestie a iaculis at erat pellentesque. Sed id semper risus in hendrerit gravida rutrum. Justo eget magna fermentum iaculis eu. Tortor condimentum lacinia quis vel eros donec ac.

\end{document}
