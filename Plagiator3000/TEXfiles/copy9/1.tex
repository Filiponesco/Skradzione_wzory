\documentclass{article}
\usepackage[utf8]{inputenc}

\begin{document}

Quisque non tellus orci ac auctor augue mauris augue. Commodo sed egestas egestas fringilla phasellus faucibus scelerisque. Accumsan tortor posuere ac ut consequat semper. Ut sem \begin{math}
	R_w=\frac{n^2-1}{n^2+2} \cdot \frac{1}{p}
\end{math}.\newline

Scelerisque viverra mauris in aliquam sem fringilla ut. Ultricies tristique $ {n \choose k}=\frac{n!}{k!(n-k)} $.\newline

Nibh mauris cursus mattis molestie a iaculis \( a^2-b^2=(a-b)(a+b) \).\newline

Viverra nam libero justo laoreet sit. Egestas dui id ornare arcu odio ut.
\[ a^3-b^3=(a-b)(a^2-ab+b^2) \]

Libero volutpat sed cras ornare arcu dui vivamus arcu felis. Faucibus turpis in eu mi bibendum neque egestas congue. Cursus metus aliquam eleifend mi in nulla posuere.

$$ S_{xy}=\sum_{i=1}^{n}x_i\cdot y_i $$

Sapien nec sagittis aliquam malesuada bibendum arcu. Velit egestas dui id ornare arcu odio ut sem. 

\begin{displaymath}
	S_x=\sum_{i=1}^{n}x_i
\end{displaymath}

Feugiat in fermentum posuere urna. Feugiat pretium nibh ipsum consequat nisl vel pretium. Aliquet eget sit amet tellus cras adipiscing enim. Massa enim nec dui nunc mattis enim ut tellus elementum.

\begin{equation}
	x_1=\frac{-b-\sqrt{\Delta}}{2a},x_2=\frac{-b+\sqrt{\Delta}}{2a}
\end{equation}

\end{document}