\documentclass{article}
\usepackage[utf8]{inputenc}

\begin{document}
Lorem ipsum dolor sit amet, consectetur adipiscing elit, sed do eiusmod tempor incididunt ut labore et dolore magna aliqua. Senectus et netus et malesuada fames ac. Et sollicitudin 
\begin{math}
	\lim_{n \to \infty}\sum_{k=1}^n \frac{1}{k^2}= \frac{\pi^2}{6}
\end{math}
.\newline
 
Nunc aliquet bibendum enim facilisis gravida neque convallis. Ipsum dolor sit amet
 $ \int\limits_{x\in Z}\! x^{n}\, dx $
 .\newline
    
Odio euismod lacinia at quis risus sed. Ullamcorper velit sed ullamcorper morbi 
\( \prod_{i=1}^ni=n! \)
.\newline

Diam vulputate ut pharetra sit amet aliquam id diam. Diam maecenas ultricies mi eget mauris pharetra et. In est ante in nibh. Duis convallis convallis tellus id interdum velit laoreet id donec. 
\[ a^{\frac{m}{n}}=\sqrt[n]{a^{m}} \]

Quam adipiscing vitae proin sagittis nisl rhoncus mattis rhoncus urna. Tellus pellentesque eu tincidunt tortor aliquam. Sed sed risus pretium quam vulputate dignissim suspendisse in est. Duis at tellus at urna. Viverra vitae congue eu consequat ac. Diam sit amet nisl suscipit adipiscing bibendum est. Lacinia quis vel eros donec ac odio tempor. Nunc faucibus a pellentesque sit amet porttitor eget dolor. Id aliquet lectus proin nibh nisl condimentum id venenatis.
$$ a^{-n}=\frac{1}{a^{n}} $$

Amet volutpat consequat mauris nunc congue nisi. Sit amet massa vitae tortor. Elementum curabitur vitae nunc sed velit dignissim sodales. 
\begin{displaymath}
	(a+b)^{2}=a^{2}+2ab+b^{2}
\end{displaymath}

Lacus laoreet non curabitur gravida arcu ac tortor dignissim. Commodo nulla facilisi nullam vehicula ipsum a arcu cursus. Tortor pretium viverra suspendisse potenti nullam ac tortor vitae. 

\begin{equation}
	(a-b)^{3}=a^{3}-3a^{2}b+3ab^{2}-b^{3}
\end{equation}

\end{document}