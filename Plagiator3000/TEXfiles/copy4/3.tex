\documentclass{article}
\usepackage[utf8]{inputenc}

\begin{document}

Ultricies mi quis hendrerit \begin{math}
	F=q\left ( E+v \times B \right )
\end{math}.\newline

Aliquet enim tortor at auctor. Massa ultricies mi quis hendrerit dolor $ a^{-n}=\frac{1}{a^{n}} $.\newline

Ut tellus elementum sagittis vitae et leo. Feugiat scelerisque varius morbi enim. Morbi tincidunt ornare massa \( f\left ( a \right ) = \frac{1}{2\Pi i} \oint \frac{f\left ( z \right )}{z-a} dz \).\newline

Pharetra sit amet aliquam id diam. Justo donec enim diam vulputate ut pharetra sit amet.

\[ a^{-\frac{m}{n}}=\frac{1}{\sqrt[n]{a^{m}}} \]

Nam libero justo laoreet sit amet cursus sit amet dictum. Ipsum faucibus vitae aliquet nec ullamcorper sit amet risus.

$$ \log _{a}(x\cdot y)=\log _{a}x+\log _{a}y $$

Vitae semper quis lectus nulla at volutpat diam ut venenatis.

\begin{displaymath}
	a^{3}+1=(a+1)(a^{2}-a+1)
\end{displaymath}

Semper viverra nam libero justo. Porta nibh venenatis cras sed felis eget velit aliquet sagittis.

\begin{equation}
	F_{G}=G\frac{m_1\cdot m_2}{r^2}
\end{equation}

\end{document}